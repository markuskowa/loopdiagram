\documentclass[12pt,letterpaper]{article}
\usepackage[utf8]{inputenc}
\usepackage{graphicx}
\usepackage{loopdiagram}

\title{The loopdiagram package}
\author{Markus Kowalewski}
\begin{document}
\maketitle

\section{Introduction}
This package provides a simple interface to create common types of spectroscopic
diagrams. All drawing macros are implemented by using the PSTricks packages.
Use either plain LaTeX or XeLaTex to compile documents. With PDFLatex there are
some PSTricks specific problems. The package is simply called by specifing
\verb|\usepackage{loopdiagram}| in the document header.

\section{Types of Diagrams}
So far three different types of diagrams are implemented: Simple energy level
diagrams, ladder diagrams, and loop diagrams. For each type of diagram a separate
environment is specified, which draws the basic structures. The interactions are
added through simple macros.

The environments for loop and ladder diagrams are defined in the same way:
\begin{verbatim}
\begin{loopdiagram}{<height>}{<ket level>}{<bra level>}
\end{verbatim}
\begin{verbatim}
\begin{ladderdiagram}{<height>}{<ket level>}{<bra level>}
\end{verbatim}   
The \verb|height| parameter defines the vertical length (or time). Furthermore,
the initial state is defined by the bra and ket symbols. These are set into math
mode automatically.
The following macros are available to set interactions in the diagrams. These
macros are designed to take care of the label placements for field labels, time
labels, and the state change.
\begin{verbatim}
\ketin{<time>}{<field label>}{<time label>}{<new state>}
\brain{<time>}{<field label>}{<time label>}{<new state>}
\end{verbatim}
These macros draw in going arrows on the ket and bra side respectively as they
are typically used to symbolize an absorptive dipole interactions. The
\verb|time| parameter determines the y-value where arrow tips the vertical lines
of the diagram.
Dipole emission interactions are given by the respective \verb|out| version of these
commands:
\begin{verbatim}
\ketout{<time>}{<field label>}{<time label>}{<new state>}
\braout{<time>}{<field label>}{<time label>}{<new state>}
\end{verbatim}
Off-resonant Raman type interactions can be drawn with the \verb|inout| commands:
\begin{verbatim}
\ketinout{<time>}{<f label in>}{<f label out>}{<t label>}{<new state>}
\brainout{<time>}{<f label in>}{<f label out>}{<t label>}{<new state>}
\end{verbatim}
The time in the intermediate state is set to zero and thus only one time label
and final state needs to be specified here.

For energy level diagrams the \verb|leveldiagram| environment is available:
\begin{verbatim}
\begin{leveldiagram}{<width>}{<height>}
\end{verbatim}
Energy levels are placed with:
\begin{verbatim}
\putlevel{<E>}{<x>}{<width>}{<label>}
\end{verbatim}
Where \verb|E| parameter specifies the energy value (y-value), \verb|x|
and \verb|width| control the horizontal placement, and \verb|label| puts
the level label to the left of the level.
To indicate off-resonant interactions with non further specified set of
levels the
\begin{verbatim}
\putvlevel{<E>}{<x>}{<width>}
\end{verbatim}
command is available ("virtual" levels). Vertical arrows denoting the interactions
with a field are drawn by:
\begin{verbatim}
\fieldup{<Emin>}{<Emax>}{<x>}{<label>}
\fielddown{<Emin>}{<Emax>}{<x>}{<label>}
\end{verbatim}


\section{Examples}
Here we go. Let's create a loop diagram like this:
\begin{center}
\begin{loopdiagram}{9}{g}{d}
\psgrid[subgriddiv=2](0,0)(0,0)(6,9)
\ketinout{5}{$\omega_1$}{$\omega_2$}{$\tau_6$}{c}
\brainout{6}{$\omega_1$}{$\omega_2$}{$\tau_5$}{e}
\ketout{7}{$\omega_s$}{$t$}{e}
\ketin{2}{$\omega_1$}{$\tau_1$}{e}
\brain{2}{$\omega_2$}{$\tau_2$}{e^\prime}
\ketout{3}{$\omega_3$}{$\tau_3$}{f}
\braout{3}{$\omega_4$}{$\tau_4$}{f^\prime}
\end{loopdiagram}
\end{center}
The grid in the background is just there to visualize the coordinate system.
This is the code to create it:
\begin{verbatim}
\begin{loopdiagram}{9}{g}{d}
\ketin{2}{$\omega_1$}{$\tau_1$}{e}
\brain{2}{$\omega_2$}{$\tau_2$}{e^\prime}
\ketout{3}{$\omega_3$}{$\tau_3$}{f}
\braout{3}{$\omega_4$}{$\tau_4$}{f^\prime}
\ketinout{5}{$\omega_1$}{$\omega_2$}{$\tau_6$}{c}
\brainout{6}{$\omega_1$}{$\omega_2$}{$\tau_5$}{e}
\ketout{7}{$\omega_s$}{$t$}{e}
\end{loopdiagram}
\end{verbatim}

Here is the same thing, but ploted as a ladder digram with the
\verb|ladderdiagram| environment:
\begin{center}
\begin{ladderdiagram}{9}{g}{d}
%\psgrid[subgriddiv=2](0,0)(0,0)(6,9)
\ketinout{5}{$\omega_1$}{$\omega_2$}{$\tau_6$}{c}
\brainout{6}{$\omega_1$}{$\omega_2$}{$\tau_5$}{e}
\ketout{7}{$\omega_s$}{$t$}{e}
\ketin{2}{$\omega_1$}{$\tau_1$}{e}
\brain{2}{$\omega_2$}{$\tau_2$}{e^\prime}
\ketout{3}{$\omega_3$}{$\tau_3$}{f}
\braout{3}{$\omega_4$}{$\tau_4$}{f^\prime}
\end{ladderdiagram}
\end{center}

And last but not least a level scheme:
\begin{center}
\begin{leveldiagram}{5}{7}
\putlevel{0}{0}{4}{g}
\putlevel{1}{1}{2}{e}

\fieldup{0}{5}{0.25}{$\omega_1$}
\fielddown{1}{5}{1.25}{$\omega_2$}
\putvlevel{5}{0}{1.5}

\fieldup{1}{6}{2.75}{$\omega_1$}
\fielddown{0}{6}{3.75}{$\omega_s$}
\putvlevel{6}{2.5}{1.5}
\end{leveldiagram}
\end{center}
created by:
\begin{verbatim}
\begin{leveldiagram}{5}{7}
\putlevel{0}{0}{4}{g}
\putlevel{1}{1}{2}{e}

\fieldup{0}{5}{0.25}{$\omega_1$}
\fielddown{1}{5}{1.25}{$\omega_2$}
\putvlevel{5}{0}{1.5}

\fieldup{1}{6}{2.75}{$\omega_1$}
\fielddown{0}{6}{3.75}{$\omega_s$}
\putvlevel{6}{2.5}{1.5}
\end{leveldiagram}
\end{verbatim}

\end{document}

