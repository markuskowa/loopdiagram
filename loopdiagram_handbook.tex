\documentclass[12pt,letterpaper]{article}
\usepackage[utf8]{inputenc}
\usepackage{graphicx}
\usepackage{loopdiagram}

\title{The loopdiagram package}
\author{Markus Kowalewski}
\begin{document}
\maketitle

\begin{figure}
\centering
\fbox{
\begin{loopdiagram}{9}{g}{d}
\psgrid[subgriddiv=2](0,0)(0,0)(6,9)
\ketintio{5}{$\omega_1$}{$\omega_2$}{$\tau_6$}{c}
\braintio{6}{$\omega_1$}{$\omega_2$}{$\tau_5$}{e}


\ketintout{7}{$\omega_s$}{$t$}{e}


\ketintin{2}{$\omega_1$}{$\tau_1$}{e}
\braintin{2}{$\omega_2$}{$\tau_2$}{e^\prime}

\ketintout{3}{$\omega_3$}{$\tau_3$}{f}
\braintout{3}{$\omega_4$}{$\tau_4$}{f^\prime}
\end{loopdiagram}
}
\caption{All possible macros in one loop diagram}
\label{fig:diagall}
\end{figure}

The code for fig. \ref{fig:diagall}
\begin{verbatim}
\begin{loopdiagram}{8.5}{g}{d}

\ketintio{5}{$\omega_1$}{$\omega_2$}{$\tau_6$}{c}
\braintio{6}{$\omega_1$}{$\omega_2$}{$\tau_5$}{e}

\ketintout{7}{$\omega_s$}{$t$}{e}

\ketintin{2}{$\omega_1$}{$\tau_1$}{e}
\braintin{2}{$\omega_2$}{$\tau_2$}{e^\prime}

\ketintout{3}{$\omega_3$}{$\tau_3$}{f}
\braintout{3}{$\omega_4$}{$\tau_4$}{f^\prime}
\end{loopdiagram}
\end{verbatim}
\end{document}